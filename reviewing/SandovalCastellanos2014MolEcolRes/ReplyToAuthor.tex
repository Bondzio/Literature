\documentclass[11pt]{article}
\usepackage[total={17cm, 24cm}]{geometry}                % See geometry.pdf to learn the layout options. There are lots.
\geometry{a4paper}                   % ... or a4paper or a5paper or ... 
%\geometry{landscape}                % Activate for for rotated page geometry
%\usepackage[parfill]{parskip}    % Activate to begin paragraphs with an empty line rather than an indent
\usepackage{graphicx}
\usepackage{amssymb}
\usepackage{epstopdf}
\usepackage[normalem]{ulem}
\usepackage{natbib}
\DeclareGraphicsRule{.tif}{png}{.png}{`convert #1 `dirname #1`/`basename #1 .tif`.png}

\usepackage{color}
\usepackage[
	bookmarks = true,
	bookmarksnumbered = false, 	% true means bookmarks in
							% navigation window are numbered
	bookmarksopen = false, 		% true means only level 1
							% are displayed.
	colorlinks = true,			% false for frames around links, true for color
	linkcolor = myred,
	citecolor = mygreen,
	urlcolor = myblue
	]{hyperref}
	
\definecolor{mygreen}{rgb}{0, 0.5, 0}	% less intense green
\definecolor{myblue}{rgb}{0, 0, 0.75}		% less intense blue
\definecolor{myred}{rgb}{0.75, 0, 0}		% less intense red
\definecolor{myrev1}{rgb}{1, 0, 0}		% less intense red; after approval, simply turn into black
\definecolor{temphidden}{rgb}{0, 0, 0}		% less intense red; to re-highlight items that finally need to be changed,
									% but should be black temporarilly
% \definecolor{myrev1}{rgb}{0, 0, 0}
\definecolor{mycorr1}{rgb}{0, 0, 1}		% less intense red; after approval, simply turn into black



% Define customised list environments.
\newenvironment{my_description}
{\begin{description}
  \setlength{\itemsep}{2pt}
  \setlength{\parskip}{0pt}
  \setlength{\parsep}{0pt}}
{\end{description}}

\newenvironment{my_enumerate}
{\begin{enumerate}
  \setlength{\itemsep}{2pt}
  \setlength{\parskip}{0pt}
  \setlength{\parsep}{0pt}}
{\end{enumerate}}

\newcommand{\ra}{$\rightarrow$\ }
\newcommand{\C}{\textbf{C:}\ }
\newcommand{\Q}{\textbf{Q:}\ }
\newcommand{\R}{\textbf{R:}\ }
\newcommand{\V}{\textbf{S:}\ }

\newcommand{\fst}{$F_{\mathrm{ST}}\ $}
\newcommand{\qst}{$Q_{\mathrm{ST}}\ $}


\title{Reply to Sandoval--Castellanos ``Kernel-beta: A novel non-parametric method for model choice in approximate Bayesian computation''}
%\author{Simon Aeschbacher}
\date{1 August 2014}                                           % Activate to display a given date or no date

\begin{document}
\maketitle
%\section{}
%\subsection{}

\section{Summary}
This paper addresses the important challenge of model selection in settings where likelihoods are difficult or impossible to compute directly. It proposes a modification to existing regression-type ABC procedures. The core ideas are to i) weight the accepted simulated data points before regression according to a kernel, and to ii) consider the estimated model likelihoods as realisations of a multinomial sampling. Point i) is a direct and obvious analogy to what has been standard for a long time in ABC parameter estimation. Point ii) is intended to provide an idea of the statistical error associated with the point estimate of the ABC-type Bayes factor. While there is some appeal in the idea behind point ii), the paper fails to provide convincing evidence in favour of the proposed approach. It also lacks a theoretical justification for the chosen implementation of point ii). Based on the current description of the analyses and results, I could not establish if this is just a matter of sub-optimal presentation and writing, or of a deeper methodological flaw. For several reasons explained below, this paper does not seem appropriate for publication at its current stage.

\section{General issues}

\begin{my_enumerate}
	\item The paper suffers from poor English writing. Oftentimes, it is not clear whether a sentence describes a fact, a hypothesis, or a speculation. Parts of the introduction and discussion seem tangential to the actual problem.
	\item The core problem with ABC-type model comparison is the lack of sufficient summary statistics. This topic is ignored here, though \cite[e.g.][]{Robert:2011fk}. I realise that this is not the main subject of the article, but am worried that it is much more important than the contribution this paper tries to make.
	\item This being a methodological paper, the description of the rationale and procedure would profit from a more formal, mathematical presentation. For instance, phrases like ``the posterior probability function of the model likelihoods vector'' are correct, but potentially confusing.
	\item It was not clear to me what exactly the values in Tables 1 and 3 meant. In Table 3, how can statistical power and error rates be larger than 1?
	\item The proposed kernel-beta method is called ``assumption-free'' several times. Yet, the choice of a kernel, the use of the Euclidean distance as metric, and a multinomial error model imply potentially strong assumptions.
	\item I am worried that the `numerical issues' encountered in the context of the logistic-regression ABC procedure were due to a flawed implementation. Unfortunately, it was not possible to me to verify this given the material provided.
\end{my_enumerate}
	

\section{Specific comments}

\subsection{Abbreviations used}
\begin{my_description}
	\item[Q] Question % \Q
	\item[C] Comment % \C
	\item[S] Suggestion % \V
	\item[R] Re-formulation or change needed (usually followed by a suggestion) % \R
	\item[\ra] Suggested change/correction
\end{my_description}

\subsection{Abstract}

\begin{my_description}
	\item[l.22] ``The model choice\dots'' \ra Model choice\dots
	\item[l.22--23] \C ``It is performed\dots'' It is not clear what this sentence is meant to say.
	\item[l.23] ``The model choice\dots'' \ra Model choice\dots
	\item[l.24] \C ``due to" repeated after l.23. \C Strictly speaking, ``absence of likelihood estimation" in ABC is not correct. While explicit computation of likelihoods is avoided, likelihoods are implicitly computed by the ABC algorithm and can in principle be retrieved from the prior and posterior distributions up to a constant.
	\item[l.25] \R ``However, the available\dots are few." \ra However, few methods are available for performing model choice by ABC.
	\item[l.26] \V ``Here I suggest\dots" \ra Here, I suggest\dots
	\item[l.29--30] \R ``This kernel-beta method altogether with\dots one real dataset." \ra This kernel-based method was compared to  a direct approach and a logistic regression using five simulated and one real dataset.
	\item[l.31] ``loses" \ra losses
	\item[l.34] ``whit" \ra with
	\item[l.36--37] This last sentence seems tangential to the main focus (see points i and ii in the Summary above), given that no specific recommendations are given in the paper.
\end{my_description}

\subsection{Introduction}
\begin{my_description}
	\item[l.43] ``The model choice, the adoption of\dots'' \ra Model choice, i.e.\ the adoption of\dots
	\item[l.49] \R Delete ``, also known as likelihood-free inference,". I know that this term is present in the literature, but it should not (see comment to l.24 above). \V ``one of the most successful family\dots" \ra one of the most widely used families\dots
	\item[l.50] \V ``absence of likelihood estimation steps" \ra absence of explicit likelihood-estimation steps. \C This sentence is at risk of glossing over the practical and conceptual challenges that ABC comes with.
	\item[l.52--57]  \C The references here seem selective and incomplete.
	\item[l.54] \V ``pathogens spread rates" \ra rates of pathogen spread
	\item[l.58] ``ABC consists in performing\dots" \ra ABC performs
	\item[l.62] ``conditional to\dots" \ra conditional on\dots
	\item[l.63] \V Delete ``analyses"
	\item[l.65] ``conditioned to\dots" \ra conditional on\dots. \V ``They are also approached from\dots" \ra Bayes factors are also computed based on\dots
	\item[l.70] \V Use correct dashes consistently (here, the en dash surrounded by spaces is appropriate). \C References to choice of summary statistics in ABC are incomplete \cite[e.g.][]{Wegmann:2009sf,Nunes:2010fk,Fearnhead:2011uq,Aeschbacher:2011fk,Blum:2012fk}
	\item[l.73--78] \C This sentence is too long.
	\item[l.74] ``alternatives" \ra methods. \C The ``direct approach'' has not yet been defined. \V ``a direct approach, where acceptance proportions are taken at face value as model likelihoods"
	\item[l.78] ``right" \ra correct
	\item[l.80--82] \C A problem here is that the correct metric for measuring this distance in unknown.
	\item[l.83] ``conditioned to" \ra conditioned on
	\item[83--85] This idea is not new and has been standard in ABC for a long time \cite[][]{Beaumont:2002bh}. The innovation proposed here is to apply this principle to model comparison, rather than just to parameter estimation.
	\item[91--94] \C This sentence is too long.
	\item[l.92] ``leading" \ra lead
	\item[l.96] \C ``It could be robust to high dimensionality problems\dots" is a statement that does not sound convincing to me.
	\item[l.97] \V Insert a comma after ``assumptions".
	\item[l.98] ``also put on test by\dots" \ra tested against
\end{my_description}



\subsection{Materials and Methods}

\begin{my_description}
	\item[l.103] ``of the parameters estimation'' \ra of parameter estimation
	\item[l.104--105] ``So, in first instance I pursuit\dots" \ra Here, I first pursue\dots
	\item[l.105] The approach is not free of assumptions (see general comment above).
	\item[l.106] \V ``Secondly,\dots" \ra Second,\dots
	\item[l.108] \V Delete ``possible".
	\item[l.109---110] Delete ``of which the model choice analysis is part".
	\item[l.111] \V ``So, the method described below\dots" \ra ``The method described below\dots"
	\item[l.112] \C Taking the Euclidean distance corresponds to the assumption of multidimensional Gaussian distribution of data points.
	\item[l.117] \R ``As a Bayesian method, ABC addresses a model choice problem by\dots" \ra Approximate Bayesian computation addresses model choice by\dots
	\item[l.118] ``gets respect another\dots" \ra ``has with respect to another\dots"
	\item[l.119] \V ``\dots the observed data, and is given\dots" \ra \dots the observed data. It is given\dots
	\item[l.127] \C ``Mixture'' instead of ``admixture"?
	\item[l.132] \V Insert comma after ``proposed here".
	\item[l.133] ``evaluated in" \ra evaluated at. \C As mentioned before, this idea is not new. It has been widely used in ABC parameter estimation. I am not sure if it has been used in ABC-type model choice, though. Please clarify.
	\item[l.135] ``So, the\dots" \ra "The\dots"
	\item[l.136] \V Delete the colon after ``by".
	\item[l.137] ``which values is" \ra evaluating to
	\item[l.139] ``evaluated in\dots" \ra evaluated at
	\item[l.141] ``sensitive" \ra crucial; ``done" \ra made
	\item[l.141--142] These three points refute the claim of an `assumption-free' approach
	\item[l.144] ``already some predilection" \ra been used before
	\item[l.147--148] \C But how to choose the rejection distance first of all?
	\item[l.148--149] \C The mere fact that the Euclidean distance has been widely used before does not guarantee that it ``could be safely used here". \V Insert comma after ``ABC".
	\item[l.152] Delete ``of".
	\item[l.154] ``So this problem is a Bayesian estimation\dots" \ra This amounts to a Bayesian estimation\dots
	\item[l.157--158] This sentence is redundant to the previous one.
	\item[l.158] \V Insert comma after ``sampling".
	\item[l.158--161] This sentence is too long.
	\item[l.159] Insert comma after ``sampling size".
	\item[l.160] \V ``\dots categories, so the larger\dots" \ra \dots categories. Therefore, the larger\dots
	\item[l.161] Insert comma after ``numbers". \Q What do you mean by ``righteously"?
	\item[l.162] ``equaling" \ra equating
	\item[l.163] Insert comma after ``estimations".
	\item[l.163--166] This sentence is too long. \V ``kernel function, which\dots" \ra ``kernel function. This\dots".
	\item[l.163--167] The relevance of ``would" and ``should" as compared to ``is" and ``are" is unclear. Are these speculations or facts?
	\item[l.166] \V Insert comma after ``distances".
	\item[l.167] ``not weighted, multinomial sample" \ra unweighted multinomial sample.
	\item[l.158--169] \C The purpose of these lines is not clear to me.
	\item[l.170] ``Now, since the BF are\dots" \ra Since the BFs are\dots. Insert comma after ``likelihoods". \Q Why ``could"? Is the statement justified or not?
	\item[l.171--172] Insert commas after ``case" and ``distribution".
	\item[l.172--175] ``Notice that this procedure\dots weighted by their probabilities:" \ra Notice that this procedure yields a posterior distribution of model likelihoods, not a point estimate. To obtain a point estimate, we may target the expected ratio of likelihoods as\dots. \C In the equation, why is there a ratio of beta pdfs? The rightmost expression is not equal to the preceding one. \R Put the comma at the end of the equation, not at the start of the following line of text (l.178).
	\item[l.176] $\beta_1$ \ra $\beta_{i1}$
	\item[l.177] Replace semi-colon by comma, and delete space after $\beta_{i1}$.
	\item[l.178] ``the Epanechnikov kernel\dots summary statistics ($s^{\prime}$)." \ra as defined above.
	\item[l.180] \C Although I presume the answer $\beta_{i1}/\beta_{j1}$ is correct, this is not what I obtained by evaluating the equation after l.175. Please verify this.
	\item[l.188] ``the way" \ra how
	\item[l.190] ``The algorithm where the simulations\dots consist in the next steps:" \ra The algorithm for independent simulations of each model consist of the following steps:
	\item[l.191--210] \C Use of index $i$ do denote association to model $i$ is inconsistent. Add it everywhere or tell the user that you omit it for simplicity.
	\item[l.191] \C A ``reference table" has not been defined.
	\item[l.198] \C Estimating a quantity by defining it seems strange to me. \V Use proper minus signs consistently (en dash instead of hyphen).
	\item[l.202] ``punctual estimate" \ra point estimate. \C This mistake occurs several times, and I am not going to repeat myself. \Q Given that $\beta_{2} = 1 - \beta_{1}$, isn't $p$ just equal to $\beta_1$? \R Insert comma before `` while".
	\item[l.204] Delete ``(simulation of random deviates from a beta pdf is trivial)". \R ``estimations" \ra estimates \C This mistake occurs multiple times and I am not going to repeat myself.
	\item[l.212] ``In such case, the sample would be\dots" \ra In this case, the sample is\dots
	\item[l.213] ``conjugate posterior of a Dirichlet pdf were the $m$ parameters would be estimated\dots" \ra conjugate posterior a Dirichlet pdf. The $m$ parameters are estimated\dots
	\item[l.214] ``could" \ra are
	\item[l.218] Insert comma after ``software".
	\item[l.225] \C ``parameter $D$ of one model as the sum\dots" \ra parameter $D_i$ of model $i$ as the sum. \C Please use the index $i$ consistently (see previous comment).
	\item[l.228] ``the $D$ parameters\dots" \ra the parameters $D_i$\dots
	\item[l.230] ``parameters being the (unnormalized) $D$ parameters of the model." \ra parameters equal to the $D_i$ on their original scale. \C I was confused by the brackets around ``unnormalized". Is normalisation now essential or not?
	\item[l.231] ``is $D_i^{\ast}$ while joint or\dots" \ra is $D_i^{\ast}$. Joint or\dots
	\item[l.233] ``pdf's respectively" \ra pdfs, respectively
	\item[l.237] ``getting" \ra obtaining
	\item[l.243] ``consisted in" \ra consisted of. \C This mistake occurs several time and I am not going to repeat myself.
	\item[l.245] ``instrumental" \ra auxiliary (?)
	\item[l.248] ``See Figure 1 for a schematic\dots" \ra Figure 1 gives a schematic\dots
	\item[l.249--250] ``with the specifics of the priors employed" \ra for the priors used
	\item[l.254] ``closest simulations to the\dots" \ra simulations closest to the\dots
	\item[l.255] ``perform a model choice" \ra ``perform model choice"
	\item[l.262] ``In it, the regression parameters were simulated\dots" \ra Regression parameters were simulated\dots
	\item[l.265] ``used to get" \ra used to obtain
	\item[l.273] \V Insert comma after ``analyses".
	\item[l.284] \Q What exactly is meant by a ``statistical group"?
\end{my_description}


\subsection{Results}
\begin{my_description}
	\item[l.297--304] \R Please revise this paragraph to improve the formulation.
	\item[l.300] ``did not showed\dots" \ra did not show; ``respect each other" \ra ``with respect to each other"
	\item[l.301] ``were more similar between them than with the logistic regression" \ra were more similar among each other than to the logistic regression\dots
	\item[l.302] \C Use either singular or plural in ``the differences was more than subtle"
	\item[l.305] \V Delete ``estimations of the". \R ``also showed differences" \ra also produced different results.
	\item[l.306] ``picking" \ra choosing
	\item[l.307--308] ``Regarding the rates of\dots were observed." \ra Error rates and statistical power were similar among the three methods, but the following tendencies were observed. \Q What exactly is meant by ``rates of error"? Is it the proportion of cases where the true model did not have the strongest support? How exactly was power defined?
	\item[l.309] \V ``rates of error" \ra error rates
	\item[l.313] ``Noticeably" \ra Notably
	\item[l.314] \Q What exactly do you mean by ``failed estimations"?
	\item[l.317--324] \C I am very concerned that the numerical issues described here might be a consequence of a flawed implementation. Was this checked against a standard routine, e.g.\ the glm package in R? Since one of the main conclusions of this paper hinges on the apparently weak performance of the logistic regression, I consider this check essential.
	\item[l.320] Delete ``developed for being error proof".
\end{my_description}


\subsection{Discussion}
\begin{my_description}
	\item[l.327--329] \C I do not think this follows from the paper, and am not able to verify this, given the lack of a thorough description of the analyses (see comments to tables below and to the apparent numerical issues with the logistic regression).
	\item[l.330] ``Direct approach\dots" \ra The direct approach\dots
	\item[l.331--332] ``accompanied of marginal loses of\dots" \ra accompanied by marginal losses of\dots
	\item[l.333] \C ``demise'' does not seem appropriate here.
	\item[l.333-347] \Q These two paragraphs seems tangential to the main focus of the paper, and are far too long. The observation of a trade-off between type-I error and statistical power is well established, so is the fact that it depends on the specific question whether low error or high power is more desirable.
	\item[l.348--351] \R Please provide evidence for the reason for the observed numerical issues with the logistic regression. Then revise this paragraph accordingly.
	\item[l.355--356] ``lack of assumption fulfillment" \ra ``violation of assumptions". \C Again, the kernel-beta approach is not free of assumptions.
	\item[l.356--357] See previous comments on the numerical issues.
	\item[l.360] \Q What is meant by ``the approaching error"?
	\item[l.361] ``full Bayesian" \ra fully Bayesian
	\item[l.362] \V ``\dots error estimates, but also its\dots \ra \dots error estimates. Moreover, its\dots
	\item[l.364] \Q See previous comments about the lack of assumptions.
	\item[l.369] ``potential of greatly improve\dots" \ra potential of greatly improving\dots; ``right" \ra appropriate
	\item[l.374] \R ``Francois" \ra Fran\c{c}ois (this also applies to the list of references)
	\item[l.375] ``focussing in\dots" \ra focussing on\dots
	\item[l.378] ``keep an eye in" \dots keep an eye on. \C What do you mean by ``computer efficiency of new developments"?
\end{my_description}
	
\section{References}
Please revise the formatting. Check abbreviations of journal names for correctness and consistency. Replace ``Francois" by ``Fran\c{c}ois", and be consistent with usage of lower- and uppercase letters in titles of articles.

\section{Data Accessibility}
Please improve the formulation. The current description reads as if the contribution is insignificant.

\section{Figures and tables}
\begin{my_description}
	\item[l.472--476] \C This sentence is too long. \R ``instead" \ra instead of.
	\item[l.477--478] \Q I was wondering why you did not include the full models given in panels H--K in your testing procedure. This would give an idea of the performance under the real-world application.
	\item[Figure 1] \Q In the main text, there should be a short verbal description and motivation of the models. Why were they chosen? Have they been previously studied? What are the challenges (e.g.\ which parameters are particularly hard to estimate)? \C In panel G, what does it mean when a population does not reach the present? Has it gone extinct? Has it been sampled or not? When?
	\item[Table 1] Caption: ``among every iteration" \ra across iterations; ``also among the different" \ra across different. \C It is unclear whether the values in the table are absolute or relative differences. Please also state that they are on the natural scale (not on a logarithmic one) if appropriate.
	\item[Table 3] Caption: \Q Why type I/II error rate? Please clarify. \R ``the naked number while error appears in" \ra accompanied by the error in. ``the total number simulations" \ra the total number of simulations; ``estimation --i.e.\ without numerical errors-)" \ra estimation, i.e. without numerical errors). \C It was not clear to me what the values in the table actually meant. How can error rates and statistical power be larger than 1? Also, is the first column the true model?
\end{my_description}


\bibliographystyle{/Users/Simon/Documents/Literature/Reviewing/FountainEtAl2013MolEcol/genetics}
\bibliography{/Users/Simon/Documents/Literature/BibDesk/Jshort_dot,/Users/Simon/Documents/Literature/BibDesk/central}
\end{document}  